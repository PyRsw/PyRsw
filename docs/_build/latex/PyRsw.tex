% Generated by Sphinx.
\def\sphinxdocclass{report}
\documentclass[letterpaper,10pt,english]{sphinxmanual}
\usepackage[utf8]{inputenc}
\DeclareUnicodeCharacter{00A0}{\nobreakspace}
\usepackage{cmap}
\usepackage[T1]{fontenc}
\usepackage{babel}
\usepackage{times}
\usepackage[Bjarne]{fncychap}
\usepackage{longtable}
\usepackage{sphinx}
\usepackage{multirow}

\addto\captionsenglish{\renewcommand{\figurename}{Fig. }}
\addto\captionsenglish{\renewcommand{\tablename}{Table }}
\floatname{literal-block}{Listing }



\title{PyRsw Documentation}
\date{November 10, 2015}
\release{0.1}
\author{PyRsw Team}
\newcommand{\sphinxlogo}{}
\renewcommand{\releasename}{Release}
\makeindex

\makeatletter
\def\PYG@reset{\let\PYG@it=\relax \let\PYG@bf=\relax%
    \let\PYG@ul=\relax \let\PYG@tc=\relax%
    \let\PYG@bc=\relax \let\PYG@ff=\relax}
\def\PYG@tok#1{\csname PYG@tok@#1\endcsname}
\def\PYG@toks#1+{\ifx\relax#1\empty\else%
    \PYG@tok{#1}\expandafter\PYG@toks\fi}
\def\PYG@do#1{\PYG@bc{\PYG@tc{\PYG@ul{%
    \PYG@it{\PYG@bf{\PYG@ff{#1}}}}}}}
\def\PYG#1#2{\PYG@reset\PYG@toks#1+\relax+\PYG@do{#2}}

\expandafter\def\csname PYG@tok@gd\endcsname{\def\PYG@tc##1{\textcolor[rgb]{0.63,0.00,0.00}{##1}}}
\expandafter\def\csname PYG@tok@gu\endcsname{\let\PYG@bf=\textbf\def\PYG@tc##1{\textcolor[rgb]{0.50,0.00,0.50}{##1}}}
\expandafter\def\csname PYG@tok@gt\endcsname{\def\PYG@tc##1{\textcolor[rgb]{0.00,0.27,0.87}{##1}}}
\expandafter\def\csname PYG@tok@gs\endcsname{\let\PYG@bf=\textbf}
\expandafter\def\csname PYG@tok@gr\endcsname{\def\PYG@tc##1{\textcolor[rgb]{1.00,0.00,0.00}{##1}}}
\expandafter\def\csname PYG@tok@cm\endcsname{\let\PYG@it=\textit\def\PYG@tc##1{\textcolor[rgb]{0.25,0.50,0.56}{##1}}}
\expandafter\def\csname PYG@tok@vg\endcsname{\def\PYG@tc##1{\textcolor[rgb]{0.73,0.38,0.84}{##1}}}
\expandafter\def\csname PYG@tok@m\endcsname{\def\PYG@tc##1{\textcolor[rgb]{0.13,0.50,0.31}{##1}}}
\expandafter\def\csname PYG@tok@mh\endcsname{\def\PYG@tc##1{\textcolor[rgb]{0.13,0.50,0.31}{##1}}}
\expandafter\def\csname PYG@tok@cs\endcsname{\def\PYG@tc##1{\textcolor[rgb]{0.25,0.50,0.56}{##1}}\def\PYG@bc##1{\setlength{\fboxsep}{0pt}\colorbox[rgb]{1.00,0.94,0.94}{\strut ##1}}}
\expandafter\def\csname PYG@tok@ge\endcsname{\let\PYG@it=\textit}
\expandafter\def\csname PYG@tok@vc\endcsname{\def\PYG@tc##1{\textcolor[rgb]{0.73,0.38,0.84}{##1}}}
\expandafter\def\csname PYG@tok@il\endcsname{\def\PYG@tc##1{\textcolor[rgb]{0.13,0.50,0.31}{##1}}}
\expandafter\def\csname PYG@tok@go\endcsname{\def\PYG@tc##1{\textcolor[rgb]{0.20,0.20,0.20}{##1}}}
\expandafter\def\csname PYG@tok@cp\endcsname{\def\PYG@tc##1{\textcolor[rgb]{0.00,0.44,0.13}{##1}}}
\expandafter\def\csname PYG@tok@gi\endcsname{\def\PYG@tc##1{\textcolor[rgb]{0.00,0.63,0.00}{##1}}}
\expandafter\def\csname PYG@tok@gh\endcsname{\let\PYG@bf=\textbf\def\PYG@tc##1{\textcolor[rgb]{0.00,0.00,0.50}{##1}}}
\expandafter\def\csname PYG@tok@ni\endcsname{\let\PYG@bf=\textbf\def\PYG@tc##1{\textcolor[rgb]{0.84,0.33,0.22}{##1}}}
\expandafter\def\csname PYG@tok@nl\endcsname{\let\PYG@bf=\textbf\def\PYG@tc##1{\textcolor[rgb]{0.00,0.13,0.44}{##1}}}
\expandafter\def\csname PYG@tok@nn\endcsname{\let\PYG@bf=\textbf\def\PYG@tc##1{\textcolor[rgb]{0.05,0.52,0.71}{##1}}}
\expandafter\def\csname PYG@tok@no\endcsname{\def\PYG@tc##1{\textcolor[rgb]{0.38,0.68,0.84}{##1}}}
\expandafter\def\csname PYG@tok@na\endcsname{\def\PYG@tc##1{\textcolor[rgb]{0.25,0.44,0.63}{##1}}}
\expandafter\def\csname PYG@tok@nb\endcsname{\def\PYG@tc##1{\textcolor[rgb]{0.00,0.44,0.13}{##1}}}
\expandafter\def\csname PYG@tok@nc\endcsname{\let\PYG@bf=\textbf\def\PYG@tc##1{\textcolor[rgb]{0.05,0.52,0.71}{##1}}}
\expandafter\def\csname PYG@tok@nd\endcsname{\let\PYG@bf=\textbf\def\PYG@tc##1{\textcolor[rgb]{0.33,0.33,0.33}{##1}}}
\expandafter\def\csname PYG@tok@ne\endcsname{\def\PYG@tc##1{\textcolor[rgb]{0.00,0.44,0.13}{##1}}}
\expandafter\def\csname PYG@tok@nf\endcsname{\def\PYG@tc##1{\textcolor[rgb]{0.02,0.16,0.49}{##1}}}
\expandafter\def\csname PYG@tok@si\endcsname{\let\PYG@it=\textit\def\PYG@tc##1{\textcolor[rgb]{0.44,0.63,0.82}{##1}}}
\expandafter\def\csname PYG@tok@s2\endcsname{\def\PYG@tc##1{\textcolor[rgb]{0.25,0.44,0.63}{##1}}}
\expandafter\def\csname PYG@tok@vi\endcsname{\def\PYG@tc##1{\textcolor[rgb]{0.73,0.38,0.84}{##1}}}
\expandafter\def\csname PYG@tok@nt\endcsname{\let\PYG@bf=\textbf\def\PYG@tc##1{\textcolor[rgb]{0.02,0.16,0.45}{##1}}}
\expandafter\def\csname PYG@tok@nv\endcsname{\def\PYG@tc##1{\textcolor[rgb]{0.73,0.38,0.84}{##1}}}
\expandafter\def\csname PYG@tok@s1\endcsname{\def\PYG@tc##1{\textcolor[rgb]{0.25,0.44,0.63}{##1}}}
\expandafter\def\csname PYG@tok@gp\endcsname{\let\PYG@bf=\textbf\def\PYG@tc##1{\textcolor[rgb]{0.78,0.36,0.04}{##1}}}
\expandafter\def\csname PYG@tok@sh\endcsname{\def\PYG@tc##1{\textcolor[rgb]{0.25,0.44,0.63}{##1}}}
\expandafter\def\csname PYG@tok@ow\endcsname{\let\PYG@bf=\textbf\def\PYG@tc##1{\textcolor[rgb]{0.00,0.44,0.13}{##1}}}
\expandafter\def\csname PYG@tok@sx\endcsname{\def\PYG@tc##1{\textcolor[rgb]{0.78,0.36,0.04}{##1}}}
\expandafter\def\csname PYG@tok@bp\endcsname{\def\PYG@tc##1{\textcolor[rgb]{0.00,0.44,0.13}{##1}}}
\expandafter\def\csname PYG@tok@c1\endcsname{\let\PYG@it=\textit\def\PYG@tc##1{\textcolor[rgb]{0.25,0.50,0.56}{##1}}}
\expandafter\def\csname PYG@tok@kc\endcsname{\let\PYG@bf=\textbf\def\PYG@tc##1{\textcolor[rgb]{0.00,0.44,0.13}{##1}}}
\expandafter\def\csname PYG@tok@c\endcsname{\let\PYG@it=\textit\def\PYG@tc##1{\textcolor[rgb]{0.25,0.50,0.56}{##1}}}
\expandafter\def\csname PYG@tok@mf\endcsname{\def\PYG@tc##1{\textcolor[rgb]{0.13,0.50,0.31}{##1}}}
\expandafter\def\csname PYG@tok@err\endcsname{\def\PYG@bc##1{\setlength{\fboxsep}{0pt}\fcolorbox[rgb]{1.00,0.00,0.00}{1,1,1}{\strut ##1}}}
\expandafter\def\csname PYG@tok@mb\endcsname{\def\PYG@tc##1{\textcolor[rgb]{0.13,0.50,0.31}{##1}}}
\expandafter\def\csname PYG@tok@ss\endcsname{\def\PYG@tc##1{\textcolor[rgb]{0.32,0.47,0.09}{##1}}}
\expandafter\def\csname PYG@tok@sr\endcsname{\def\PYG@tc##1{\textcolor[rgb]{0.14,0.33,0.53}{##1}}}
\expandafter\def\csname PYG@tok@mo\endcsname{\def\PYG@tc##1{\textcolor[rgb]{0.13,0.50,0.31}{##1}}}
\expandafter\def\csname PYG@tok@kd\endcsname{\let\PYG@bf=\textbf\def\PYG@tc##1{\textcolor[rgb]{0.00,0.44,0.13}{##1}}}
\expandafter\def\csname PYG@tok@mi\endcsname{\def\PYG@tc##1{\textcolor[rgb]{0.13,0.50,0.31}{##1}}}
\expandafter\def\csname PYG@tok@kn\endcsname{\let\PYG@bf=\textbf\def\PYG@tc##1{\textcolor[rgb]{0.00,0.44,0.13}{##1}}}
\expandafter\def\csname PYG@tok@o\endcsname{\def\PYG@tc##1{\textcolor[rgb]{0.40,0.40,0.40}{##1}}}
\expandafter\def\csname PYG@tok@kr\endcsname{\let\PYG@bf=\textbf\def\PYG@tc##1{\textcolor[rgb]{0.00,0.44,0.13}{##1}}}
\expandafter\def\csname PYG@tok@s\endcsname{\def\PYG@tc##1{\textcolor[rgb]{0.25,0.44,0.63}{##1}}}
\expandafter\def\csname PYG@tok@kp\endcsname{\def\PYG@tc##1{\textcolor[rgb]{0.00,0.44,0.13}{##1}}}
\expandafter\def\csname PYG@tok@w\endcsname{\def\PYG@tc##1{\textcolor[rgb]{0.73,0.73,0.73}{##1}}}
\expandafter\def\csname PYG@tok@kt\endcsname{\def\PYG@tc##1{\textcolor[rgb]{0.56,0.13,0.00}{##1}}}
\expandafter\def\csname PYG@tok@sc\endcsname{\def\PYG@tc##1{\textcolor[rgb]{0.25,0.44,0.63}{##1}}}
\expandafter\def\csname PYG@tok@sb\endcsname{\def\PYG@tc##1{\textcolor[rgb]{0.25,0.44,0.63}{##1}}}
\expandafter\def\csname PYG@tok@k\endcsname{\let\PYG@bf=\textbf\def\PYG@tc##1{\textcolor[rgb]{0.00,0.44,0.13}{##1}}}
\expandafter\def\csname PYG@tok@se\endcsname{\let\PYG@bf=\textbf\def\PYG@tc##1{\textcolor[rgb]{0.25,0.44,0.63}{##1}}}
\expandafter\def\csname PYG@tok@sd\endcsname{\let\PYG@it=\textit\def\PYG@tc##1{\textcolor[rgb]{0.25,0.44,0.63}{##1}}}

\def\PYGZbs{\char`\\}
\def\PYGZus{\char`\_}
\def\PYGZob{\char`\{}
\def\PYGZcb{\char`\}}
\def\PYGZca{\char`\^}
\def\PYGZam{\char`\&}
\def\PYGZlt{\char`\<}
\def\PYGZgt{\char`\>}
\def\PYGZsh{\char`\#}
\def\PYGZpc{\char`\%}
\def\PYGZdl{\char`\$}
\def\PYGZhy{\char`\-}
\def\PYGZsq{\char`\'}
\def\PYGZdq{\char`\"}
\def\PYGZti{\char`\~}
% for compatibility with earlier versions
\def\PYGZat{@}
\def\PYGZlb{[}
\def\PYGZrb{]}
\makeatother

\renewcommand\PYGZsq{\textquotesingle}

\begin{document}

\maketitle
\tableofcontents
\phantomsection\label{index::doc}


Contents:


\chapter{plan}
\label{plan:welcome-to-pysw-s-documentation}\label{plan::doc}\label{plan:plan}
List or problems to solve
\begin{enumerate}
\item {} 
solve 1d 1-layer SW (periodic/boundaries)

\item {} 
solve 1d n-layer SW

\item {} 
solve 2d 1-layer SW

\item {} 
solve 2d n-layer SW

\item {} 
Linear Stability Calculations

\end{enumerate}

Equations and latex to include:
\begin{enumerate}
\item {} 
derivation of SW

\item {} 
governing equations for different models (see above)

\item {} 
different numerical methods:
\begin{enumerate}
\item {} 
FD Sardouney

\item {} 
spectral

\item {} 
WENO

\item {} 
f2py

\item {} 
openmp

\item {} 
openmp + mpi

\end{enumerate}

\end{enumerate}


\chapter{One-Layer Rotating Shallow Water Model}
\label{sw_intro::doc}\label{sw_intro:one-layer-rotating-shallow-water-model}

\section{Classical Form}
\label{sw_intro:classical-form}
The one-layer, two-dimensional rotating shallow water model on a
rotating \(f\)-plane can be written as,
\begin{gather}
\begin{split}\begin{aligned}
\frac{\partial u}{\partial t} + \left(\bf{u} \cdot \nabla\right) u - f v
&= - g \frac{\partial h}{\partial x},\\
\frac{\partial v}{\partial t} + \left(\bf{u} \cdot \nabla\right) v + f u
&= - g \frac{\partial h}{\partial y},\\
\frac{\partial h}{\partial t} + \nabla \cdot \left( h \bf{u} \right) &= 0.\end{aligned}\end{split}\notag
\end{gather}
This describes the motion of a pancake like fluid in that it is thin in
the vertical and much longer in the horizontal. It contains pressure
forces due to the free surface and a Coriolis pseudo-force because of
the rotating frame of reference.

The fluid moves as columns that can be translated in the horizontal and
stretched/contracted in the vertical. If the height of a column changes
then the voriticty must change, as can be reflected in the fact that, in
the absence of forcing and dissipation, Potential Vorticity is conserved
following the motion,
\begin{gather}
\begin{split}\frac{D}{Dt} \left(
\frac{ \frac{\partial v}{\partial x} - \frac{\partial u}{\partial y} + f }{h} \right) = 0.\end{split}\notag
\end{gather}

\section{Conservation Form}
\label{sw_intro:conservation-form}
There are a variety of forms, these equations can be written. One of
them is conservation where the three fields that have time derivatives
are \(U = hu\), \(V = hv\) and \(h\).
\begin{gather}
\begin{split}\begin{aligned}
\frac{\partial U}{\partial t}
+ \frac{\partial}{\partial x}\left( \frac{U^2}{h} + \frac{g h^2}{2} \right)
+ \frac{\partial}{\partial y}\left( \frac{U V}{h}  \right)
- f V
&= 0,\\
\frac{\partial V}{\partial t}
+ \frac{\partial}{\partial x}\left( \frac{U V}{h}  \right)
+ \frac{\partial}{\partial y}\left( \frac{V^2}{h} + \frac{g h^2}{2} \right)
+ f U
&= 0,\\
\frac{\partial h}{\partial t} + \nabla \cdot \left( h \bf{u} \right) &= 0.\end{aligned}\end{split}\notag
\end{gather}
Conservation form is attractive because it can help to ensure that,
using a clever numerical scheme, some quantitites are conserved. Note
that if you have topography then there are source terms that appear on
the right-hand side and this causes some problems.


\section{Vorticity-Bernoulli Form}
\label{sw_intro:vorticity-bernoulli-form}
A third form arises from rewriting the nonlinear acceleration terms as a
gradient and a cross product term. Using a vector identity, it can be
shown that the above system is mathematically equivalent to the
following,
\begin{gather}
\begin{split}\begin{aligned}
\frac{\partial u}{\partial t}  - q h v
&= - \frac{\partial B}{\partial x},\\
\frac{\partial v}{\partial t}  + q h u
&= - \frac{\partial B}{\partial y},\\
\frac{\partial h}{\partial t} + \nabla \left( h \bf{u} \right) &= 0.\end{aligned}\end{split}\notag
\end{gather}
Note that above we have defined the vorticity and the Bernoulli
function,
\begin{gather}
\begin{split}\begin{aligned}
q &= \frac{\zeta + f}{h}
 = \frac{\frac{\partial v}{\partial x} - \frac{\partial u}{\partial y} + f}{h},\\
B & = g h + \frac12 \left( u^2 + v^2 \right).\end{aligned}\end{split}\notag
\end{gather}
Before we look at solving this complicated set of equations we consider
the one and a half dimensional limit.


\section{\(1 \frac{1}{2}\) Dimensional Limit}
\label{sw_intro:dimensional-limit}
We assume that none of the variables depend on one horizontal direction,
say \(y\). But, it is very important to realize that the velocity in
that direction is not necessarily zero. Indeed, if you have flow in the
\(x\)-direction, the Coriolis force will deflect it to the right,
which will then generate a flow that is perpendicular. This will
continue and often give rise to inerital oscialltions in the horizontal.
So we can have motion in either direction but the motion only changes
with respect to \(x\).

If we simply the governing equations we get
\begin{gather}
\begin{split}\begin{aligned}
\frac{\partial u}{\partial t}  - q h v
&= - \frac{\partial B}{\partial x},\\
\frac{\partial v}{\partial t}  + q h u
&= 0,\\
\frac{\partial h}{\partial t} + \frac{\partial}{\partial x} \left( h u \right) &= 0.\end{aligned}\end{split}\notag
\end{gather}
where the vorticity and the Bernoulli function simplify to,
\begin{gather}
\begin{split}\begin{aligned}
q & = \frac{\zeta + f}{h} = \frac{\frac{\partial v}{\partial x} + f}{h},\\
B & = g h^2 + \frac12 \left( u^2 + v^2 \right).\end{aligned}\end{split}\notag
\end{gather}

\section{Conserved Quantities}
\label{sw_intro:conserved-quantities}
In the purely conservative or nondissipative limit, there are three
quantities that are exactly conserved.
\begin{enumerate}
\item {} 
Mass:

\end{enumerate}
\begin{gather}
\begin{split}M = \int_D h \, dA\end{split}\notag
\end{gather}\begin{enumerate}
\setcounter{enumi}{1}
\item {} 
Total Energy: sum of potential and kinetic energies

\end{enumerate}
\begin{gather}
\begin{split}E = \frac12 \int_D \left(g h^2 + h( u^2 + v^2)\right) \, dA\end{split}\notag
\end{gather}\begin{enumerate}
\setcounter{enumi}{2}
\item {} 
Potenal Enstrophy:

\end{enumerate}
\begin{gather}
\begin{split}Q = \frac12 \int_D  h q^2 \, dA\end{split}\notag
\end{gather}
In a numerical model we cannot expect these to be conserved but we would
like them to be close to be conserved. If they are very badly conserved
than this could reflect that the numerical scheme is behaving badly.
However, just because these are conserved that does not guarantee that
the solution is correct. But they are usually good indicators as to how
our method is doing in the conservative limit.

Of course when nonconservative forces are introduced things will change.
One might argue that since the world is non-dissipative then we don’t
need to worry about conserving these. However, it is desirable to know
that basis of your model is well behaved and therefore why we should
worry about conserved quantities.


\chapter{Indices and tables}
\label{index:indices-and-tables}\begin{itemize}
\item {} 
\DUspan{xref,std,std-ref}{genindex}

\item {} 
\DUspan{xref,std,std-ref}{modindex}

\item {} 
\DUspan{xref,std,std-ref}{search}

\end{itemize}



\renewcommand{\indexname}{Index}
\printindex
\end{document}
