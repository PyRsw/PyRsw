\documentclass[11pt]{article}
\usepackage{geometry}                
\geometry{a4paper,left=2.5cm,right=2.5cm,top=2.5cm,bottom=2.5cm}
\usepackage{natbib}
\usepackage{color}
\definecolor{mygreen}{RGB}{28,172,0} % color values Red, Green, Blue
\definecolor{mylilas}{RGB}{170,55,241}
\usepackage{epsfig}
\usepackage{amssymb,amsmath}
\usepackage{enumerate}
\usepackage{enumitem}
\usepackage[utf8]{inputenc}
\usepackage{hyperref}
\usepackage{mathtools}


\newcommand{\ssd}{\text{ssd}}
\newcommand{\sS}{\mathsf{S}}
\newcommand{\tot}{\text{tot}}

\begin{document}


\newcommand{\com}{\, ,}
\newcommand{\per}{\, .}

%% Averages
% Use \bar to over line solo symbols

\newcommand{\av}[1]{\bar{#1}}
\newcommand{\avbg}[1]{\overline{#1}}
\newcommand{\avbgg}[1]{\overline{#1}}

% A nice definition
\newcommand{\defn}{\ensuremath{\stackrel{\mathrm{def}}{=}}}

% space in equations
\newcommand{\qqand}{\qquad \text{and} \qquad}
\newcommand{\qand}{\quad \text{and} \quad}

% equations
\def\beq{\begin{equation}}
\def\eeq{\end{equation}}

\def\bea{\begin{align}}
\def\ena{\end{align}}


% calculus
\newcommand{\ord}{\mathcal{O}}
\newcommand{\p}{\partial}
\newcommand{\ii}{{\rm i}}
\newcommand{\dd}{{\rm d}}
\newcommand{\id}{{\, \rm d}}
\newcommand{\ee}{{\rm e}}
\newcommand{\DD}{{\rm D}}
\newcommand{\wavy}{\text{wavy}}
\newcommand{\qg}{\text{qg}}
\newcommand{\dt}{\Delta t}
\newcommand{\dx}{\Delta x}
\newcommand{\be}{\beta}

\newcommand{\al}{\alpha}
\newcommand{\bx}{\boldsymbol{x}}
\newcommand{\by}{\boldsymbol{y}}
\newcommand{\bu}{\boldsymbol{u}}
\newcommand{\bv}{\boldsymbol{v}}


\newcommand{\half}{\tfrac{1}{2}}
\newcommand{\halfrho}{\tfrac{1}{2}}
\newcommand{\rz}{{}}
\newcommand{\bn}{\boldsymbol{\hat n}}
\newcommand{\br}{\boldsymbol{r}}
\newcommand{\bR}{\boldsymbol{R}}
\newcommand{\bA}{\ensuremath {\boldsymbol {A}}}
\newcommand{\bB}{\ensuremath {\boldsymbol {B}}}
\newcommand{\bU}{\ensuremath {\boldsymbol {U}}}
\newcommand{\bE}{\ensuremath {\boldsymbol {E}}}
\newcommand{\bN}{\ensuremath {\boldsymbol {\mathrm{N}}}}
\newcommand{\bJ}{\ensuremath {\boldsymbol {J}}}
\newcommand{\bXX}{\ensuremath {\boldsymbol {\mathcal{X}}}}
\newcommand{\bFF}{\ensuremath {\boldsymbol {F}}}
\newcommand{\bF}{\ensuremath {\boldsymbol {F}^{\sharp}}}
\newcommand{\bG}{\ensuremath {\boldsymbol G}}
\newcommand{\bSigma}{\ensuremath {\boldsymbol {\Sigma}}}
\newcommand{\bvarphi}{\ensuremath {\boldsymbol {\varphi}}}
\newcommand{\bxi}{\ensuremath {\boldsymbol {\xi}}}
\newcommand{\avbxi}{\overline{\ensuremath {\boldsymbol {\xi}}}}

% math cal

\newcommand{\J}{\mathcal{J}}
\newcommand{\K}{\mathcal{K}}
\newcommand{\cG}{\mathcal{G}}
\newcommand{\cF}{\mathcal{F}}
\newcommand{\cN}{\mathcal{N}}
\newcommand{\cL}{\mathcal{L}}


% san serif for matrices and differential operators
%\newcommand{\helmn}{\mathsf{H}_n}
\newcommand{\helmm}{\triangle_m}
\newcommand{\helmn}{\triangle_n}
\newcommand{\helms}{\triangle_s}
\newcommand{\helm}{\triangle}
\newcommand{\sA}{\mathsf{A}}
\newcommand{\sB}{\mathsf{B}}
\newcommand{\sG}{\mathsf{G}}
\newcommand{\sI}{\mathsf{I}}
\newcommand{\sJ}{\mathsf{J}}
\newcommand{\gsJ}{\breve{\mathsf{J}}}
\newcommand{\sU}{\mathsf{U}}
\newcommand{\sP}{\mathsf{P}}
\newcommand{\sQ}{\mathsf{Q}}
\newcommand{\sR}{\mathsf{R}}
\newcommand{\sL}{\mathsf{L}}
\newcommand{\Lu}{\mathsf{L}(\what{u}_k)}
\newcommand{\Nu}{\mathsf{N}(\what{u}_k)}
\renewcommand{\L}{\mathsf{L}}
\newcommand{\N}{\mathsf{N}}
\newcommand{\sH}{\mathsf{H}}
\renewcommand{\sJ}{\mathsf{J}}
\renewcommand{\sI}{\mathsf{I}}
\renewcommand{\L}{\mathsf{L}}
\newcommand{\sM}{\mathsf{M}}
\newcommand{\sT}{\mathsf{T}}
\newcommand{\sGamma}{\mathsf{\Gamma}}
\newcommand{\sOmega}{\mathsf{\Omega}}
\newcommand{\sSigma}{\mathsf{\Omega}}
\newcommand{\sbeta}{\mathsf{\beta}}
\newcommand{\sPi}{\mathsf{\Pi}}
\newcommand{\sC}{\mathsf{C}}
\newcommand{\sQy}{\mathsf{Q}}
\renewcommand{\sb}{\mathsf{b}}

% u
\newcommand{\uhat}{\what{u}_k}

% angle brackets

\def\la{\langle}
\def\ra{\rangle}
\def\laa{\left \langle}
\def\raa{\right \rangle}


%grads and div's
\newcommand{\bcdot}{\hspace{-0.1em} \boldsymbol{\cdot} \hspace{-0.12em}}
\newcommand{\bnabla}{\boldsymbol{\nabla}}
\newcommand{\bnablaH}{\bnabla_{\! \mathrm{h}}}
\newcommand{\grad}{\bnabla}
\newcommand{\gradH}{\bnablaH}
\newcommand{\curl}{\bnabla \!\times\!}
\newcommand{\diver}{\bnabla \bcdot }
\newcommand{\cross}{\times}
\newcommand{\lap}{\nabla^2}


%varthetas and thetas
\newcommand{\vth}{\vartheta}
\newcommand{\psii}{\psi^{\mathrm{i}}}
\newcommand{\thb}{\theta^{\mathrm{-}}}
\newcommand{\vthb}{\vartheta^{\mathrm{-}}}
\newcommand{\vthbhat}{{\hat{\vartheta}}^{\mathrm{-}}}
\newcommand{\vThb}{\varTheta^{\mathrm{-}}}
\newcommand{\psib}{\psi^{\mathrm{-}}}
\newcommand{\tht}{\theta^{\mathrm{+}}}
\newcommand{\vtht}{\vartheta^{\mathrm{+}}}
\newcommand{\vththat}{{\hat{\vartheta}}^{\mathrm{+}}}
\newcommand{\vthtbhat}{{\hat{\vartheta}}^{\pm}}
\newcommand{\vTht}{\varTheta^{\mathrm{+}}}
\newcommand{\vthtb}{\vartheta^{\pm}}
\newcommand{\vThtb}{\varTheta^{\pm}}

% nondimensional numbers
\renewcommand{\Re}{\mathrm{Re}}
\newcommand{\Ro}{\mathrm{Ro}}
\newcommand{\Ri}{\mathrm{Ri}}

%psi's
%Galerking coefficient for psi:
\newcommand{\gpsi}{\breve \psi}
\newcommand{\gpsic}{{\breve \psi}^\star}
\newcommand{\gtau}{\breve \tau}
\newcommand{\gtauc}{{\breve \tau}^\star}
\newcommand{\gphi}{\breve \phi}
\newcommand{\gq}{\breve q}
\newcommand{\gU}{\breve U}
\newcommand{\gQ}{\breve Q}
\newcommand{\gsigma}{\breve \sigma}


\newcommand{\psit}{\psi^{\mathrm{+}}}
\newcommand{\psithat}{{\hat{\psi}}^{\mathrm{+}}}
\newcommand{\psibhat}{{\hat{\psi}}^{\mathrm{-}}}
\newcommand{\psitb}{\psi^{\pm}}
\newcommand{\psitbhat}{{\hat{\psi}}^\pm}
\newcommand{\St}{S^{\mathrm{+}}}
\newcommand{\Sb}{S^{\mathrm{-}}}
\newcommand{\phb}{\phi^{\mathrm{-}}}
\newcommand{\pht}{\phi^{\mathrm{+}}}
\newcommand{\tautb}{\tau^{\pm}}
\newcommand{\sigmatb}{\sigma^{\pm}}


\newcommand{\bur}{\left(\tfrac{f_0}{N}\right)^2}
\newcommand{\ibur}{\left(\tfrac{N}{f_0}\right)^2}
\newcommand{\Nm}{N_{\mathrm{mix}}}
\newcommand{\xim}{\xi_{\mathrm{mix}}}
\newcommand{\hs}{h_*}
\renewcommand{\sp}{\mathsf{p}}
\newcommand{\se}{\mathsf{e}}
\newcommand{\sptb}{\mathsf{p}^\pm}


%nmax is a problem:
%\newcommand{\nmax}{n_{\mathrm{max}}}
\newcommand{\nmax}{\mathrm{N}}
\newcommand{\mmax}{\mathrm{M}}

\newcommand{\WKB}{\mathrm{WKB}}
\newcommand{\Lam}{\Lambda}
\newcommand{\tha}{\theta}
\newcommand{\kap}{\kappa}
\newcommand{\bphi}{\boldsymbol{\phi}}
\newcommand{\third}{\tfrac{1}{3}}
\newcommand{\cs}{c^\star}
\newcommand{\dstar}{{\star\star}}
\newcommand{\nt}{n^{\mathrm{trnc}}}
\newcommand{\sDp}{\mathsf{D}^1_{\nmax}}
\newcommand{\sDpp}{\mathsf{D}^2_{\nmax}}
\newcommand{\sD}{\mathsf{D}}
\newcommand{\sDN}{\mathsf{D_\nmax}}
\newcommand{\sK}{\mathsf{K_2}}
\newcommand{\stheta}{\mathsf{\theta}}
\newcommand{\sphi}{\mathsf{\phi}}
\newcommand{\sq}{\mathsf{q}}
\newcommand{\cosech}{\text{csch}\,}
\newcommand{\sinc}{\text{sinc}\,}

%%%%%%%%% %%%%
\newcommand{\zp}{z^+}
\newcommand{\zm}{z^-}
\newcommand{\qA}{q^A_{\nmax}}
\newcommand{\psiB}{\psi^B_{\nmax}}
\newcommand{\phiB}{\phi^B_{\nmax}}
\newcommand{\eye}{\boldsymbol{\hat{i}}}
\newcommand{\jay}{\boldsymbol{\hat{j}}}
\newcommand{\kay}{\boldsymbol{\hat{k}}}
\newcommand{\psiG}{\psi^{\mathrm{G}}}
\newcommand{\qG}{q^{\mathrm{G}}}
\newcommand{\uG}{u^{\mathrm{G}}}
\newcommand{\UG}{U^{\mathrm{G}}}
\newcommand{\UGN}{U^{\mathrm{G}}_{\nmax}}
\newcommand{\QGN}{Q^{\mathrm{G}}_{\nmax}}
\newcommand{\sumoddn}{\sum_{n = 1, n~ \text{odd}}^{\nmax}}

% bretherton 
\newcommand{\qBr}{q_{\mathrm{Br}}}
\newcommand{\psiBr}{\psi_{\mathrm{Br}}}

\newcommand{\ep}{\epsilon}


\section*{Layered quasigeostrophic model}

The $\nmax$-layer quasigeostrophic (QG) potential vorticity is

\begin{align}
{q_1} &= \lap\psi_1 + \frac{f_0^2}{H_1} \left(\frac{\psi_{2}-\psi_1}{g'_{1}}\right)\,,  \qquad & i =1\com \nonumber \\
{q_n} &= \lap\psi_n + \frac{f_0^2}{H_n} \left(\frac{\psi_{n-1}-\psi_n}{g'_{n-1}}  - \frac{\psi_{i}-\psi_{n+1}}{g'_{i}}\right)\,,  \qquad &i = 2,\nmax-1 \com \nonumber \\
{q_\nmax} &= \lap\psi_\nmax + \frac{f_0^2}{H_\nmax} \left(\frac{\psi_{\textsf{N}-1}-\psi_\nmax}{g'_{\nmax-1}}\right) + \frac{f_0}{H_\nmax}h_b (x,y)\,,  \qquad & i =\nmax\,,
\end{align}
where $q_n$ is the n'th layer QG potential vorticity, and $\psi_n$ is the streamfunction, 
 $f_0$ is the inertial frequency, n'th $H_n$ is the layer depth, and $h_b$ is the 
bottom topography. (Note that in QG $h_b/H_\nmax << 1$.) Also the n'th buoyancy
jump (reduced gravity) is
\begin{equation}
g'_n \equiv g \frac{\rho_{n}-\rho_{n+1}}{\rho_n}\com
\end{equation}
where $g$ is the acceleration due to gravity and $\rho_n$ is the layer density.

The dynamics of the system is given by the evolution of PV. We introduce a background flow that can vary in the horizontal.  The streamfunction associated with this flow can be denoted with $\Psi_n(x,y)$ for each layer and geostrophy yields its corresponding velocity $\vec{V_n} = (U_n(x,y),V_n(x,y))$ where $\Psi_{ny} = - U_n$ and $\Psi_{nx} = V_n$.  We can perturb the stream function in each layer into a background flow and deviations from that flow as,
\begin{align}
\psi_n^{\tot} = \Psi_n + \psi_n.
\end{align}

%FJP: typo in Q_x somewhere 
With this basic decomposition we can than write out the corresponding decompositions in velocity
\begin{align}
\label{eq:Uequiv}
u_n^{{\tot}} = U_n - \psi_{n y}\com \nonumber \\
v_n^{\tot} = V_n + \psi_{n x} \com
\end{align}
and
\begin{equation}
q_n^{\tot} = Q_n + \delta_{n\nmax}\frac{f_0}{H_\nmax}h_b + q_n \com
\end{equation}
where $Q_n + \delta_{n\nmax}\frac{f_0}{H_\nmax}h_b$ is n'th layer background PV,
we obtain the evolution equations
\begin{align}
\label{eq:qg_dynamics}
{q_n}_t + \mathsf{J}(\psi_n,q_n + \delta_{n \nmax} \frac{f_0}{H_\nmax}h_b )& + U_n ({q_n}_x + \delta_{n \nmax} \frac{f_0}{H_\nmax}h_{bx}) + V_n ({q_n}_y + \delta_{n \nmax} \frac{f_0}{H_\nmax}h_{by})+ \nonumber
\\ & {Q_n}_y {\psi_n}_x - {Q_n}_x {\psi_n}_y = \ssd
- r_{ek} \delta_{n\nmax} \lap \psi_n \com \qquad n = 1,\nmax\com
\end{align}
where $\ssd$ is 
stands for small scale dissipation, which is achieved by an spectral exponential filter
or hyperviscosity, and $r_{ek}$ is the linear bottom drag coefficient. The Dirac delta,
$\delta_{nN}$, indicates that the drag is only applied in the bottom layer.

% FJP: is topography counted twice?

\subsection*{Linear Stability Analysis}

In order to study the stability of a jet in the context of our $n$-layer QG model we focus our attention on basic states that consist of zonal flows. i.e. $\Psi_n(y)$ only.  If we assume that the quadratic quantities  we can then linearize to obtain in the conservative limit over a flat bottom,
\begin{align}
\label{eq:qglin_dynamics}
{q_n}_t  + U_n {q_n}_x + {Q_n}_y {\psi_n}_x  = 0,
\end{align}
for $n = 1, \cdots, N$.

We assume that the perturbations are normal modes in the zonal direction and time,
$$
\psi_n  = \Re[ \hat \psi_n e^{i(kx - \omega t)} ].
$$
This implies that the PV will be modified appropriately and we denote it with $\hat q_n$.

We substitute this into the linear equations and then divide by the exponential to obtain,
\begin{align}
%\label{eq:qglin_dynamics}
c  {\hat q_n}  =  U_n {\hat q_n} +  {Q_n}_y {\hat \psi_n} ,
\end{align}
where the basic state only depends on $y$, and layer of course, and we have introduced the phase speed $c=\omega/k$.  Note that the actual PVs are
\begin{align}
{\hat q_1} &= (\partial_{yy} - k^2) \hat \psi_1 + \frac{f_0^2}{H_1} \left(\frac{\hat \psi_{2}-\hat \psi_1}{g'_{1}}\right)\,,  \qquad & i =1\com \nonumber \\
{\hat q_n} &= (\partial_{yy} - k^2)\psi_n + \frac{f_0^2}{H_n} \left(\frac{\hat \psi_{n-1}-
\hat \psi_n}{g'_{n-1}}  - \frac{\hat \psi_{i}-\hat \psi_{n+1}}{g'_{i}}\right)\,,  \qquad &i = 2,\nmax-1 \com \nonumber \\
{\hat q_\nmax} &= (\partial_{yy} - k^2)\hat \psi_\nmax + \frac{f_0^2}{H_\nmax} \left(\frac{\hat \psi_{\textsf{N}-1} - \hat \psi_\nmax}{g'_{\nmax-1}}\right)\,,  \qquad & i =\nmax\,,
\end{align}

\section*{Special case: one-layer model}

In the one-layer case we have
\begin{align}
%\label{eq:qglin_dynamics}
c  {\hat q_1}  =  U_1 {\hat q_1} +  {Q_1}_y {\hat \psi_1} ,
\end{align}
\begin{align}
{\hat q_1} = \left[ \partial_{yy} - k^2 -  \frac{f_0^2}{g'_1 H_1} \right] \hat \psi_1.
\end{align}

\section*{Special case: two-layer model}

In the two-layer case we have
\begin{align}
%\label{eq:qglin_dynamics}
c  {\hat q_n}  =  U_n {\hat q_n} +  {Q_n}_y {\hat \psi_n} ,
\end{align}

\begin{align}
{\hat q_1} &= \left[ \partial_{yy} - k^2 -  \frac{f_0^2}{g'_1 H_1}\right] \hat \psi_1 + \frac{f_0^2}{g'_1 H_1} \hat \psi_{2}, \\
{\hat q_2} &= \frac{f_0^2}{g'_1 H_2}\hat \psi_1+ \left[ \partial_{yy} - k^2 -  \frac{f_0^2}{g'_1 H_2} \right] \hat \psi_2 
.
\end{align}

\end{document}

\section*{Special case: two-layer model}
With $\nmax = 2$, an alternative notation for the perturbation of potential vorticities can be written as
\begin{align}
    q_1 &= \lap \psi_1 + F_1 (\psi_2 - \psi_1) \nonumber\\
    q_2 &= \lap \psi_2 + F_2 (\psi_1  - \psi_2)\com
\end{align}
where we use the following definitions
where
\begin{equation}
F_1 \equiv \frac{k_d^2}{1 + \delta^2}\,, \qquad \:\:\text{and} \qquad F_2 \equiv \delta \,F_1\,,
\end{equation}
with the deformation wavenumber
\begin{equation}
k_d^2 \equiv \, \frac{f_0^2}{g} \frac{H_1+H_2}{H_1 H_2} \per
\end{equation}
With this notation, the ``stretching matrix'' is simply
\begin{equation}
\sS = \begin{bmatrix}
- F_1 \qquad \:\:\:\:F_1\\
F_2 \qquad -  + F_2
\end{bmatrix}\per
\end{equation}
The inversion relationship in Fourier space is
\begin{equation}
\begin{bmatrix}
\hat{\psi}_1\\
\hat{\psi}_2\\
\end{bmatrix}
= \frac{1}{\text{det} \: \sB}
\begin{bmatrix}
-(\kappa^2 + F_2) \qquad \:\:\:\:-F_1\\
\:\:\:\: -F_2 \qquad - (\kappa^2 + F_1)
\end{bmatrix}
\begin{bmatrix}
\hat{q}_1\\
\hat{q}_2\\
\end{bmatrix}\com
\end{equation}
where 
\begin{equation}
\qquad \text{det}\, \sB = \kappa^2\left(\kappa^2 + F_1 + F_2\right)\,.
\end{equation}


\end{document}



