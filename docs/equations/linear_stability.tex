\documentclass[11pt]{article}
\usepackage{geometry}                
\geometry{a4paper,left=2.5cm,right=2.5cm,top=2.5cm,bottom=2.5cm}
\usepackage{natbib}
\usepackage{color}
\definecolor{mygreen}{RGB}{28,172,0} % color values Red, Green, Blue
\definecolor{mylilas}{RGB}{170,55,241}
\usepackage{epsfig}
\usepackage{amssymb,amsmath}
\usepackage{enumerate}
\usepackage{enumitem}
\usepackage[utf8]{inputenc}
\usepackage{hyperref}
\usepackage{mathtools}


\newcommand{\ssd}{\text{ssd}}
\newcommand{\sS}{\mathsf{S}}
\newcommand{\tot}{\text{tot}}

\usepackage[procnames]{listings}
\usepackage{color}

\begin{document}

\definecolor{keywords}{RGB}{255,0,90}
\definecolor{comments}{RGB}{0,0,113}
\definecolor{red}{RGB}{160,0,0}
\definecolor{green}{RGB}{0,150,0}

\lstset{language=Python, 
        basicstyle=\ttfamily\small, 
        keywordstyle=\color{keywords},
        commentstyle=\color{comments},
        stringstyle=\color{red},
        showstringspaces=false,
        identifierstyle=\color{green},
        procnamekeys={def,class}}
 

\input{symbols}

\section*{Linear Stability Analysis}

\subsection*{One-Layer Shallow Water}

In this subsection we consider the one-layer reduced gravity RSW model with topography below.  
We define the following:
\begin{itemize}
\item $H$: mean depth of the layer
\item $z=\eta$: height of the free surface
\item $z=-H + \eta_B$: height of the topography.
\item $h = H + \eta - \eta_B$: total depth of layer
\item $(u,v)$: horizontal velocity
\item $g'$: reduced gravity
\item $\rho_0$: reference density
\end{itemize}
The governing nonlinear equations are,
\begin{align*}
\frac{\partial u}{\partial t} + {\vec u} \cdot \vec \nabla u - f v & 
%= -g \frac{\partial \eta}{\partial x} 
= - g \frac{\partial}{\partial x} \left( h + \eta_B \right) , \\
 \frac{\partial v}{\partial t} + {\vec u} \cdot \vec \nabla v + f u & 
%= -g \frac{\partial \eta}{\partial y} 
= - g \frac{\partial}{\partial y} \left( h + \eta_B \right) , \\
\frac{\partial h}{\partial t} + \vec\nabla \cdot \left( h \vec u_1 \right) & = 0.
\end{align*}

\subsection*{Basic State}

To study shear flows in a meridional channel we consider solutions of the form,
\begin{align*}
u & = U_B(y), \\
v & = 0,\\
h & = H_B(y).
\end{align*}
For this to be an exact solution we require that the flow is in geostrophic balance,
$$
f U_B = - g \frac{d}{dy}\left( H_B + \eta_B \right).
$$
\subsection*{Perturbation}

We perturb the basic state with infinitesimal quantities,
\begin{align*}
u & = U_B(y) + u', \\
v & = 0     + v',\\
h & = H_B(y) + h'.
\end{align*}

We substitute our perturbation into the governing equations and drop the primes (for brevity) and cancelling out the geostrophic terms 
\begin{align*}
\frac{\partial u}{\partial t} + (u + U_B) \frac{\partial u}{\partial x} + v \frac{\partial}{\partial y} \left(u + U_B \right)  - f v & = - g \frac{\partial h}{\partial x}, \\
 \frac{\partial v}{\partial t}   + (u + U_B) \frac{\partial v}{\partial x} + v \frac{\partial v }{\partial y} + f u 
 & = - g \frac{\partial h}{\partial y}, \\
\frac{\partial h}{\partial t}  + (u + U_B) \frac{\partial h}{\partial x}   + v \frac{\partial}{\partial y}(H_B + h)
+ (H_B + h)  & \left( \frac{\partial u}{\partial x} + \frac{\partial v}{\partial y} \right) =  0.
\end{align*}
Now we neglect the quadratic terms to obtain the linearized equations,
\begin{align*}
\frac{\partial u}{\partial t}
& = - U_B \frac{\partial u}{\partial x} + \left( f - \frac{d U_B}{d y}  \right) v  - g \frac{\partial h_1}{\partial x}, \\
 \frac{\partial v}{\partial t}    & =  - f u  - U_B \frac{\partial v}{\partial x}  - g \frac{\partial h_1}{\partial y}, \\
\frac{\partial h}{\partial t}   & = - H_B \frac{\partial u}{\partial x}    - v \frac{d H_B}{d y}
 - H_B \frac{\partial v}{\partial y} - U_B \frac{\partial h}{\partial x} .
\end{align*}

Finally, we assume a normal mode decomposition in the zonal direction and time,
\begin{align}
[u, v, h] = \mbox{Re}\left\{ e^{ik(x - c t)} [\hat u, ik \hat v, \hat h] \right\},
\end{align}
which we can substitute into the above equations to yield in the inviscid limit
\begin{align*}
c \hat u  &  = U_B \hat u - (f - \frac{dU_B}{dy}) \hat v + g \hat h, \\
c \hat v  &  =- \frac{f}{k^2} \hat u + U_B \hat v -\frac{ g}{k^2} \frac{d h}{d y}, \\
c \hat h  &=   H_B \hat u  +  \frac{d}{d y}\left(H_B \hat v\right) +  U_B \hat h.
\end{align*}

\end{document}

We assume that the bottom topography is located at $z = \eta_B$, the mean depth of the lower layer is $H_2^M$, the internal interface deformations are described by $z = H_2^M + \eta_2$, the mean depth of the upper layer is $H_1^M$ and the depth fo the free surface is $z = H_1^M + H_2^M + \eta_1$.  
Our governing equations can be written as
\begin{align*}
\frac{\partial u_1}{\partial t} + {\vec u_1} \cdot \vec \nabla u_1 - f v_1 & 
= -g \frac{\partial \eta_1}{\partial x} 
= - g \frac{\partial}{\partial x} \left( h_1 + h_2 + \eta_B \right) , \\
 \frac{\partial v_1}{\partial t} + {\vec u_1} \cdot \vec \nabla v_1 + f u_1 & 
= -g \frac{\partial \eta_1}{\partial y} 
= - g \frac{\partial}{\partial y} \left( h_1 + h_2 + \eta_B \right) , \\
\frac{\partial h_1}{\partial t} + \vec\nabla \cdot \left( h_1 \vec u_1 \right) & = 0, \\
\frac{\partial u_2}{\partial t} + {\vec u_2} \cdot \vec \nabla u_2 - f v_2 & 
= -g \frac{\partial \eta_1}{\partial x} - g' \frac{\partial \eta_2}{\partial x} 
= -  \frac{\partial}{\partial x} \left( g h_1 + (g + g') (h_2 + \eta_B)\right) , \\
\frac{\partial v_2}{\partial t} + {\vec u_2} \cdot \vec \nabla v_2+ f u_2 & 
= -g \frac{\partial \eta_1}{\partial y} - g' \frac{\partial \eta_2}{\partial y} 
= -  \frac{\partial}{\partial y} \left( g h_1 + (g + g') (h_2 + \eta_B)\right) , \\
\frac{\partial h_2}{\partial t} + \vec\nabla \cdot \left( h_2 \vec u_2 \right) & = 0.
\end{align*}
The relation between the interface displacements and the layer depths are,
\begin{align*}
h_1 & = H_1^M  + \eta_1 - \eta_2, \\
h_2 & = H_2^M + \eta_2 - \eta_B.
\end{align*}
Note that the reduced gravity between the two active layers is 
$$ 
g' = \frac{\rho_2 - \rho_1}{\rho_1} g.
$$

\subsection{Perturbation}

We perturb the basic state in the standard manner and allow for the possibility of an active lower layer since this was observed to be the case in the laboratory experiments,
\begin{align*}
u_1 & = U_1(y) + u_1', \\
v_1 & = 0     + v_1',\\
h_1 & = H_1(y) + h_1', \\
u_2 & = U_2(y) + u_2', \\
v_2 & = 0     + v_2',\\
h_2 & = H_2(y) + h_2'.
\end{align*}

We substitute our perturbation into the governing equations and drop the primes (for brievity) and cancelling out the geostrophic terms 
\begin{align*}
\frac{\partial u_1}{\partial t} + (u_1 + U_1) \frac{\partial u_1}{\partial x} + v_1 \frac{\partial}{\partial y} \left(u_1 + U_1\right)  - f v_1 
& = - g \frac{\partial}{\partial x}(h_1 + h_2)+ \nu \nabla^2_H u_1, \\
 \frac{\partial v_1}{\partial t}   + (u_1 + U_1) \frac{\partial v_1}{\partial x} + v_1 \frac{\partial v_1 }{\partial y} + f u_1 
 & = - g \frac{\partial}{\partial y}(h_1 + h_2)+ \nu \nabla^2_H u_1, \\
\frac{\partial h_1}{\partial t}  + (u_1 + U_1) \frac{\partial h_1}{\partial x}   + v_1 \frac{\partial}{\partial y}(H_1 + h_1)
+ &(H_1 + h_1)  \left( \frac{\partial u_1}{\partial x} + \frac{\partial v_1}{\partial y} \right)  = \kappa \nabla^2_H uh_1, \\
\frac{\partial u_2}{\partial t} + (u_2 + U_2) \frac{\partial u_2}{\partial x} + v_2 \frac{\partial}{\partial y} \left(u_2 + U_2\right)  - f v_2 & 
= - g \frac{\partial}{\partial x}\left( \frac{\rho_1}{\rho_2} h_1  + h_2\right)+ \nu \nabla^2_H u_2, \\
\frac{\partial v_2}{\partial t} + (u_2 + U_2) \frac{\partial v_2}{\partial x} + v_2 \frac{\partial v_2 }{\partial y} + f u_2 & 
= - g \frac{\partial}{\partial y}\left( \frac{\rho_1}{\rho_2} h_1 + h_2\right)+ \nu \nabla^2_H v_2, \\
\frac{\partial h_2}{\partial t}  + (u_2 + U_2) \frac{\partial h_2}{\partial x}   + v_2 \frac{\partial}{\partial y}(H_2 + h_2)
+ &(H_2 + h_2)  \left( \frac{\partial u_2}{\partial x} + \frac{\partial v_2}{\partial y} \right)  = \kappa \nabla^2_H h_2.
\end{align*}
Now we neglect the quadratic terms to obtain the linearized equations,
\begin{align*}
\frac{\partial u_1}{\partial t} + U_1 \frac{\partial u_1}{\partial x} + v_1 \frac{d U_1}{d y}  - f v_1 
& = - g \frac{\partial}{\partial x}\left(h_1 + h_2\right) + \nu \nabla_H^2 u_1, \\
 \frac{\partial v_1}{\partial t}   + U_1 \frac{\partial v_1}{\partial x}  + f u_1  & = - g \frac{\partial}{\partial y}\left(h_1 + h_2\right) + \nu \nabla_H^2 v_1, \\
\frac{\partial h_1}{\partial t}  + U_1 \frac{\partial h_1}{\partial x}   + v_1 \frac{\partial H_1}{\partial y}
+ &H_1 \left( \frac{\partial u_1}{\partial x} + \frac{\partial v_1}{\partial y} \right)  =  \kappa \nabla_H^2 h_1, \\
\frac{\partial u_2}{\partial t} + U_2 \frac{\partial u_2}{\partial x} + v_2 \frac{d U_2}{d y}  - f v_2 & 
=  - g \frac{\partial}{\partial x}\left( \frac{\rho_1}{\rho_2} h_1  + h_2\right) + \nu \nabla_H^2 u_2,\\
\frac{\partial v_2}{\partial t} + U_2 \frac{\partial v_2}{\partial x} + f u_2 & 
=  - g \frac{\partial}{\partial y}\left( \frac{\rho_1}{\rho_2} h_1  + h_2\right)+ \nu \nabla_H^2 v_2, \\
\frac{\partial h_2}{\partial t}  + U_2 \frac{\partial h_2}{\partial x}   + v_2 \frac{\partial H_2}{\partial y}
+ &H_2 \left( \frac{\partial u_2}{\partial x} + \frac{\partial v_2}{\partial y} \right)  =  \kappa \nabla_H^2 h_2.
\end{align*}

Therefore, we can combine the equations and get the following
\begin{align*}
\frac{\partial u_1}{\partial t}  &  =-  U_1 \frac{\partial u_1}{\partial x} + \left( f  - \frac{dU_1}{dy} \right) v_1 - g \frac{\partial}{\partial x}(h_1 + h_2) + \nu \nabla_H^2 u_1, \\
\frac{\partial v_1}{\partial t}  &  = - U_1 \frac{\partial v_1}{\partial x} - f u_1- g \frac{\partial}{\partial y}(h_1 + h_2)+ \nu \nabla_H^2 v_1, \\
\frac{\partial h_1}{\partial t}  &= - U_1 \frac{\partial h_1}{\partial x}   - v_1 \frac{\partial H_1}{\partial y}
- H_1 \left( \frac{\partial u_1}{\partial x} + \frac{\partial v_1}{\partial y} \right) + \kappa \nabla_H^2 h_1 , \\
\frac{\partial u_2}{\partial t}  & =   - U_2 \frac{\partial u_2}{\partial x}  +  \left( f  - \frac{dU_2}{dy} \right) v_2 
-  g \frac{\partial}{\partial x}\left( \frac{\rho_1}{\rho_2} h_1  + h_2\right) + \nu \nabla_H^2 u_2, \\
\frac{\partial v_2}{\partial t}  & = - U_2 \frac{\partial v_2}{\partial x} - f u_2 
-  g \frac{\partial}{\partial y}\left( \frac{\rho_1}{\rho_2} h_1  + h_2\right)+ \nu \nabla_H^2 v_2, \\
\frac{\partial h_2}{\partial t}  &= - U_2 \frac{\partial h_2}{\partial x}   - v_2 \frac{\partial H_2}{\partial y}
- H_2 \left( \frac{\partial u_2}{\partial x} + \frac{\partial v_2}{\partial y} \right)  + \kappa \nabla_H^2 h_2.
\end{align*}

Finally, we assume a normal mode decomposition in the zonal direction and time,
\begin{align}
[u_1, v_1, \eta_1, u_2, v_2, \eta_2] = \mbox{Re}\left\{ e^{ik(x - c t)} [\hat u_1, ik \hat v_1, \hat \eta_1, \hat u_2,  ik \hat v_2, \hat \eta_2] \right\},
\end{align}
which we can substitute into the above equations to yield in the inviscid limit
\begin{align*}
c \hat u_1  &  = U_1 \hat u_1 - (f - \frac{dU_1}{dy}) \hat v_1 + g (\hat h_1 + \hat h_2), \\
c \hat v_1  &  =- \frac{f}{k^2} \hat u_1 + U_1 \hat v_1 -\frac{ g}{k^2} \frac{d}{d y}(\hat h_1 + \hat h_2), \\
c \hat h_1  &=   H_1 \hat u_1  +  \frac{d}{d y}\left(H_1 \hat v_1\right) +  U_1 \hat h_1,\\
c \hat u_2 & =   g \frac{\rho_1}{\rho_2} \hat h_1 +  U_2 \hat u_2 - (f - \frac{dU_2}{dy}) \hat v_2  
+ g \hat h_2, \\
c \hat v_2  & = -  \frac{g}{k^2} \frac{\rho_1}{\rho2} \frac{d \hat h_1}{d y} - \frac{f}{k^2} \hat u_2 
+ U_2 \hat v_2- \frac{g}{k^2} \frac{d \hat h_2}{d y}, \\
c \hat h_2  &=  H_2 \hat u_2 + \frac{d}{d y} \left( H_2 \hat v_2 \right)  + U_2 \hat h_2.
\end{align*}
