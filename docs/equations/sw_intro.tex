\documentclass[11pt]{article}
\usepackage{geometry}                
\geometry{a4paper,left=2.5cm,right=2.5cm,top=2.5cm,bottom=2.5cm}
\usepackage{natbib}
\usepackage{color}
\definecolor{mygreen}{RGB}{28,172,0} % color values Red, Green, Blue
\definecolor{mylilas}{RGB}{170,55,241}
\usepackage{epsfig}
\usepackage{amssymb,amsmath}
\usepackage{enumerate}
\usepackage{enumitem}
\usepackage[utf8]{inputenc}
\usepackage{hyperref}
\usepackage{mathtools}


\newcommand{\ssd}{\text{ssd}}
\newcommand{\sS}{\mathsf{S}}
\newcommand{\tot}{\text{tot}}

\begin{document}


\input{symbols}

\section*{One-Layer Rotating Shallow Water Model}

\subsection*{Classical Form}

The one-layer, two-dimensional rotating shallow water model
on a rotating $f$-plane can be written as,
\begin{align} 
\frac{\partial u}{\partial t} + \left(\bf{u} \cdot \nabla\right) u - f v 
&= - g \frac{\partial h}{\partial x},\\
\frac{\partial v}{\partial t} + \left(\bf{u} \cdot \nabla\right) v + f u 
&= - g \frac{\partial h}{\partial y},\\
\frac{\partial h}{\partial t} + \nabla \cdot \left( h \bf{u} \right) &= 0.
\end{align}
This describes the motion of a pancake like fluid in that it is thin in the vertical
and much longer in the horizontal.  It contains pressure forces due to the free surface
and a Coriolis pseudo-force because of the rotating frame of reference.

The fluid moves as columns that can be
translated in the horizontal and stretched/contracted in the vertical.  If the 
height of a column changes then the voriticty must change, as can be reflected 
in the fact that, in the absence of forcing and dissipation, 
Potential Vorticity is conserved following the motion,
$$
\frac{D}{Dt} \left( 
\frac{ \frac{\partial v}{\partial x} - \frac{\partial u}{\partial y} + f }{h} \right) = 0.
$$

\subsection*{Conservation Form}

There are a variety of forms, these equations can be written.  
One of them is conservation where the three fields that have time 
derivatives are $U = hu$, $V = hv$ and $h$.  
\begin{align} 
\frac{\partial U}{\partial t} 
+ \frac{\partial}{\partial x}\left( \frac{U^2}{h} + \frac{g h^2}{2} \right)
+ \frac{\partial}{\partial y}\left( \frac{U V}{h}  \right) 
- f V 
&= 0,\\
\frac{\partial V}{\partial t} 
+ \frac{\partial}{\partial x}\left( \frac{U V}{h}  \right) 
+ \frac{\partial}{\partial y}\left( \frac{V^2}{h} + \frac{g h^2}{2} \right)
+ f U 
&= 0,\\
\frac{\partial h}{\partial t} + \nabla \cdot \left( h \bf{u} \right) &= 0.
\end{align}

Conservation form is attractive because it can help to ensure that, using a clever numerical
scheme, some quantitites are conserved.  Note that if you have topography then there
are source terms that appear on the right-hand side and this causes some problems.

\subsection*{Vorticity-Bernoulli Form}

A third form arises from rewriting the nonlinear 
acceleration terms as a gradient and a cross product term.  Using a vector identity,
it can be shown that the above system is mathematically equivalent to the following,
\begin{align} 
\frac{\partial u}{\partial t}  - q h v 
&= - \frac{\partial B}{\partial x},\\
\frac{\partial v}{\partial t}  + q h u 
&= - \frac{\partial B}{\partial y},\\
\frac{\partial h}{\partial t} + \nabla \left( h \bf{u} \right) &= 0.
\end{align}
Note that above we have defined the vorticity and the Bernoulli function,
\begin{align}
q &= \frac{\zeta + f}{h} 
 = \frac{\frac{\partial v}{\partial x} - \frac{\partial u}{\partial y} + f}{h},\\
B & = g h + \frac12 \left( u^2 + v^2 \right).
\end{align}

Before we look at solving this complicated set of equations we consider the one and a half dimensional limit.

\subsection*{$1 \frac{1}{2}$ Dimensional Limit}

We assume that none of the variables depend on one horizontal direction, say $y$. 
But, it is very important to realize that the velocity in that direction is not necessarily zero.
Indeed, if you have flow in the $x$-direction, the Coriolis force will deflect it to the right,
which will then generate a flow that is perpendicular.  This will continue and often give rise
to inerital oscialltions in the horizontal.  So we can have motion in either direction but
the motion only changes with respect to $x$.  

If we simply the governing equations we get
\begin{align} 
\frac{\partial u}{\partial t}  - q h v 
&= - \frac{\partial B}{\partial x},\\
\frac{\partial v}{\partial t}  + q h u 
&= 0,\\
\frac{\partial h}{\partial t} + \frac{\partial}{\partial x} \left( h u \right) &= 0.
\end{align}
where the vorticity and the Bernoulli function simplify to,
\begin{align}
q & = \frac{\zeta + f}{h} = \frac{\frac{\partial v}{\partial x} + f}{h},\\
B & = g h^2 + \frac12 \left( u^2 + v^2 \right).
\end{align}

\subsection*{Conserved Quantities}

In the purely conservative or nondissipative limit, 
there are three quantities that are exactly conserved.  

1) Mass:
$$
M = \int_D h \, dA
$$

2) Total Energy: sum of potential and kinetic energies
$$
E = \frac12 \int_D \left(g h^2 + h( u^2 + v^2)\right) \, dA
$$

3) Potenal Enstrophy: 
$$
Q = \frac12 \int_D  h q^2 \, dA
$$

In a numerical model we cannot expect these to be conserved 
but we would like them to be close to be conserved.  
If they are very badly conserved than this could reflect that
the numerical scheme is behaving badly.
However, just because these are conserved that does not 
guarantee that the solution is correct. 
But they are usually good indicators as to how our method is 
doing in the conservative limit.  

Of course when nonconservative forces are introduced things will change.
One might argue that since the world is non-dissipative then we don't
need to worry about conserving these.  However, it is desirable to
know that basis of your model is well behaved and therefore why
we should worry about conserved quantities.

\end{document}