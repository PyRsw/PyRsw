\documentclass[11pt]{article}
\usepackage{geometry}                
\geometry{a4paper,left=2.5cm,right=2.5cm,top=2.5cm,bottom=2.5cm}
\usepackage{natbib}
\usepackage{color}
\definecolor{mygreen}{RGB}{28,172,0} % color values Red, Green, Blue
\definecolor{mylilas}{RGB}{170,55,241}
\usepackage{epsfig}
\usepackage{amssymb,amsmath}
\usepackage{enumerate}
\usepackage{enumitem}
\usepackage[utf8]{inputenc}
\usepackage{hyperref}
\usepackage{mathtools}


\newcommand{\ssd}{\text{ssd}}
\newcommand{\sS}{\mathsf{S}}
\newcommand{\tot}{\text{tot}}

\usepackage[procnames]{listings}
\usepackage{color}

\begin{document}

\definecolor{keywords}{RGB}{255,0,90}
\definecolor{comments}{RGB}{0,0,113}
\definecolor{red}{RGB}{160,0,0}
\definecolor{green}{RGB}{0,150,0}

\lstset{language=Python, 
        basicstyle=\ttfamily\small, 
        keywordstyle=\color{keywords},
        commentstyle=\color{comments},
        stringstyle=\color{red},
        showstringspaces=false,
        identifierstyle=\color{green},
        procnamekeys={def,class}}
 

\input{symbols}

\section*{Examples}

\subsection*{Geostrophic Adjustment: 1D and 1L}

In the directory \texttt{src} you will find an example entitled \texttt{example\_1D\_1L\_spectral.py}

First, libraries are imported.  Two standard ones are numpy, for calculations, and matplotlib.pyplot for plotting.  Those are standard to numpy.  Then, there are four other things that are imported:
\begin{itemize}
\item {\bf Steppers} This contains different time-stepping functions.  At the moment we have Euler, Adams-Bashforth 2 (AB2) and Runge-Kutta 4 (RK4).  PyRsw uses adaptive time stepping to try and be more efficient in how the solution is marched forward.
\item {\bf Fluxes} This contains the fluxes for the RSW model.  At the moment there is only the option for a pseudo-spectral model but this will be generalized to include a Finite Volume method as well.
\item {\bf PyRsw} This is the main library and importing Simulation imports the core of the library.  
\item {\bf constants}  This has some useful constants, more can be added if desired.
\end{itemize}

\noindent After the libraries are imported then a simulation object is created.
\begin{lstlisting}
sim = Simulation()
\end{lstlisting}

\noindent  Below specifies the geometry in $x$ and $y$: [Options 'periodic', 'walls']

\noindent We use AB2, a spectral method: [Options: Euler, AB2, RK4]

\noindent We solve the nonlinear dynamics (can be Linear) 

\noindent  Use spectral sw model (no other choices).  Maybe hide this.
\begin{lstlisting}
sim.geomx       = 'walls'            
sim.geomy       = 'periodic'
sim.stepper     = Step.AB2       
sim.method      = 'Spectral'       
sim.dynamics    = 'Nonlinear'    
sim.flux_method = Flux.spectral_sw
\end{lstlisting}

We specify a lot of parameters.  There are some default values that are specified in {\tt PyRsw}.
\begin{lstlisting}
sim.Lx  = 4000e3          # Domain extent               (m)
sim.Ly  = 4000e3          # Domain extent               (m)
sim.geomx = 'periodic'    # Boundary Conditions
sim.geomy = 'periodic'    # Boundary Conditions
sim.Nx  = 128             # Grid points in x
sim.Ny  = 1               # Grid points in y
sim.Nz  = 1               # Number of layers
sim.g   = 9.81            # Gravity                     (m/sec^2)
sim.f0  = 1.e-4           # Coriolis                    (1/sec)
sim.cfl = 0.05            # CFL coefficient             (m)
sim.Hs  = [100.]          # Vector of mean layer depths (m)
sim.rho = [1025.]         # Vector of layer densities   (kg/m^3)
sim.end_time = 36.*hour   # End Time                    (sec)
\end{lstlisting}

We can specify the periodicity of plotting and whether we want a life animation or make a video.  More on this this later.
\begin{lstlisting}
sim.output = False        # True or False
sim.savet  = 1.*hour      # Time between saves
\end{lstlisting}

Specify periodicity of diagnostics and whether to compute them.  This is not tested.
\begin{lstlisting}
sim.diagt    = 2.*minute  # Time for output
sim.diagnose = False      # True or False
\end{lstlisting}

Initialize the simulation.
\begin{lstlisting}
sim.initialize()
\end{lstlisting}

Specify the initial conditions.  There is an option whether we want the domain in $x$ or $y$.  At the moment there is no difference because there is no $\beta$-plane but this will be added.
\begin{lstlisting}
for ii in range(sim.Nz):  # Set mean depths
    sim.soln.h[:,:,ii] = sim.Hs[ii]

# Gaussian initial conditions
x0 = 1.*sim.Lx/2.      # Centre
W  = 200.e3                # Width
amp = 1.                  # Amplitude
if sim.Ny==1:
    sim.soln.h[:,:,0] += amp*np.exp(-(sim.x-x0)**2/(W**2)).reshape((sim.Nx,1))
elif sim.Nx==1:
    sim.soln.h[:,:,0] += amp*np.exp(-(sim.y-x0)**2/(W**2)).reshape((1,sim.Ny))
\end{lstlisting}

Solve the problem.
\begin{lstlisting}
sim.run()             
\end{lstlisting}

Plot the Hovm\"oller diagram in time versus space.
\begin{lstlisting}
if sim.Ny==1:
    plt.figure               
    t = np.arange(0,sim.end_time+sim.plott,sim.plott)/86400.
        
    for L in range(sim.Nz):
        field = sim.hov_h[:,0,:].T - np.sum(sim.Hs[L:])
        plt.subplot(sim.Nz,1,L+1)
        plt.pcolormesh(sim.x/1e3,t, field,
            cmap=sim.cmap, vmin = 0, vmax = amp)
        plt.xlim([sim.x[0]/1e3, sim.x[-1]/1e3])
        plt.ylim([t[0], t[-1]])
        plt.title(r"$Hovm{\"o}ller Plot\, {of} \,\, \eta$")
        plt.xlabel(r"$distance \, \, (km)$")
        plt.ylabel(r"$Time \, \, (days)$")
        plt.colorbar()
    plt.show()
\end{lstlisting}

\begin{figure}[h]
\begin{center}
\includegraphics[width=12cm]{Figures/ex1_fig1.png}
\caption{Final solution for the test case.}
\end{center}
\end{figure}

\begin{figure}[h]
\begin{center}
\includegraphics[width=12cm]{Figures/ex1_fig2.png}
\caption{Hovm\"oller plot for the test case.}
\end{center}
\end{figure}

There is a second example called {\tt example\_1D\_geoadjust2.py} that begins with a hyperbolic tangent profile instead if a Gaussian initial condition.

\subsection*{Geostrophic Adjustment: 2D and 1L}

The basic script is almost identical to the 1D case.  The changes are as follows:
\begin{itemize}
\item[1.] Set $Nx$ and $Ny$ both equal to $128$, and from this we build a 2D grid.   
\item[2.] Define the initial conditions on a 2D grid.
\item[3.] The plotting is different.  We plot a 2D field using \tt{pcolormesh} and we don't do a Hovm\"oller plot.
\end{itemize}

\subsection*{Bickley Jet: 2D and 1L}

Following Poulin and Flierl (2003) and Irwin and Poulin (2014), we look at the instability of a jet.
\end{document}